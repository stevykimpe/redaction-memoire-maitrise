

A ce titre, une analyse automatique des radiographies permettrait d'affiner les méthodes de reconstruction {\itshape 3D} des structures osseuses tel que le {\itshape LIO} les étudie. D'autres équipes de chercheurs s'intéressent à l'analyse radiographique assistée par ordinateur afin de fournir des systèmes facilitant l'interprétation des experts médicaux, ce qui présente le bénéfice de réduire le risque de production de diagnostiques erronés.
\\

Les erreurs de diagnostiques de traumastisme de la colonne vertébrale se présentent fréquemment sur des données radiographiques ; à savoir dans $5$ à $20\%$ des cas selon \cite{Platzer2006}. Un diagnostique erroné entraine des incompréhensions de la pathologie et des retard sur les traitements. Ses conséquences cliniques sont donc non négligeables.

Ainsi, des erreurs de diagnostique sur des troubles vertébraux ont mené à de sévères complications, avec des cas de dégénérescence recensé dans certaines recherches \cite{Francesco2019, Davis1993}, telle qu'une ostéophyte (déformation osseuse) étendue.

Les complications vertébrales sont susceptibles de s'agraver et d'induire des dégénérescences neuronales ou des paraplégies. C'est en particulier le cas des traumatismes cervicaux dont l'impact sévére est immédiat.
\\

Ce chapitre présente un résumé de l'étude de la littérature en lien avec l'analyse cervicale.

Dans un premier temps, l'anatomie cervicale et les systèmes d'imagerie {\itshape 2D} et {\itshape 3D} sont décrits afin de mieux appréhender les modalités du diagnostique cervical.

Ensuite, l'état de l'art de la détection, la segmentation et l'identification cervicale est développé.

Enfin, une méthodologie à suivre pour la suite de la recherche est proposée.


\section{Contextualisation}

    L'analyse d'images médicales assistée par ordinateur présente plusieurs enjeux en soi, et c'est la raison qui motive la recherche scientifique à s'intéresser aux nouveaux apports qui peuvent être ajoutés aux méthodes actuelles.
    \\

    En dehors de l'attente d'obtenir des informations fines permettant une reconstruction {\itshape 3D} davantage précise, une extraction de la localisation et des contours vertébraux permet une automatisation des calculs des paramètres cliniques conduisant les radiologues au diagnostique de la mobilité cervicale.

    Ces informations permettraient alors de réduire le risque de mauvaises interprétations des radiographies de la part des experts médicaux, puis qu'un automatisme leur permettra d'obtenir des informations fiables cruciales au bon déroulement de l'analyse médicale.
    \\\\

    \subsection{L'impact des diagnostiques erronés}

    Selon \cite{Platzer2006}, les diagnostiques vertébraux erronés surviennent dans $5$ à $20\%$ des cas. La problématique suivante s'impose alors : les traumatismes de la colonne vertébrale mal diagnostiqués
    \subsection{Les systèmes d'imageries médicales}
        \subsubsection{L'imagerie par résonance magnétique}
        \subsubsection{La tomodensitométrie}
        \subsubsection{Les radiographies à rayons X}
        \subsubsection{Intérêt d'une reconstruction {\itshape 3D} à partir de radiographies biplanaires}
\section{Anatomie des vertèbres cervicales}
\section{Des approches de traîtement de l'image}
    \subsection{Amélioration de contraste}  % 2 algos
    \subsection{Détecteurs de marqueurs visuels}
        % Harris, GHT, Canny
\section{Des approches d'apprentissage automatique}
    \subsection{Forêt aléatoire}
    \subsection{Réseau de neurones}
    % où décrire les modèles statistiques ?









\section{Détection des vertèbres cervicales}

    La détection vertébrale est généralement traitée selon trois axes principaux.
    \begin{enumerate}
        \item Souvent proposées comme étapes de prétraitement réduisant la zone de recherche, la littérature scientifique exhibe plusieurs méthodes de détection de régions vertébrales ; à savoir des zones réduites de l'image contenant les vertèbres cibles.
        \item La littérature associe principalement la détection vertébrale à la détection de certains marqueurs vertébraux choisis arbitrairement tels que les centres ou coins vertébraux.
        \item L'état de l'art propose également des méthodes de segmentation instantanée, combinant la détection vertébrale et l'extration des contours vertébraux en une unique étape. Ces méthodes, basées sur des techniques d'apprentissage profond, sont détaillées dans la section suivante.
        \\
    \end{enumerate}

    \subsection{Localisation de régions vertébrales}

        Les méthodes initialement proposées avaient le mérite d'être simples, mais le défaut de ne pas être totalement automatique et de nécessiter l'intervention d'un utilisateur pour plusieurs tâches de prétraitement.

        La détection semi-automatique d'une région quadrilatère autour des cervicales C3 à C7 fut par exemple mise en place dans \cite{Benjelloun2006a} pour des radiographie latérales. La méthode proposée applique une fonction de modélisation des zones inter- et intra-vertébrales pour déterminer les délimitations du quadrilatère. La construction de cette fonction nécessite toutefois qu'un opérateur clique sur chaque centre vertébraux des cervicales C3 à C7. L'automatisation de cette technique est donc non optimale puisque chaque vertèbres nécessite une entrée ; cela reste acceptable pour les cinq cervicales traitées mais ne peut s'étendre aux douze vertèbres thoraciques et cinq lombaires sans une perte de temps conséquente pour l'utilisateur.
        \\

        Des méthodes totalement automatiques de détection de la région cervicale dans des radiographies en vue sagittale furent proposées une dizaine d'années plus tard dans \cite{AlArif2016a} puis \cite{AlArif2017a}.

        Le premier papier présente une méthode découpant en un premier temps l'image médicale en plusieurs patchs de mêmes tailles et orientations ; et une forêt aléatoire de Hough performe une classification binaire pour déterminer si la région d'intérêt cible est contenue en partie dans ce patch. Dans un deuxième temps, la région sélectionnée est redécoupée avec des patchs de tailles et orientations variables, permettant par la classification binaire une sélection fine de la région cervicale.

        {\itshape Al Arif et al.} proposent ensuite une méthode associant la localisation de la région cervicale avec un problème de segmentation avec une faible résolution de l'image médicale. Cette technique se base sur l'utilisation de réseaux de neurones profonds totalement convolutifs, afin d'extraire la zone cervicale de la radiographie en classant chaque pixel comme ciblé ou non. Cette approche s'appuie sur le succés grandissant de l'apprentissage profond dans l'analyse médicale assistée par ordinateur. Une fois entrainé, le réseau est capable de détecter la région d'intérêt directement à partir de l'image médicale.

        La technique initialement proposée par {\itshape Al Arif} fournit une région cible contenant les cervicales C3 à C6 dans $100\%$ des cas testés et la C7 dans $70\%$ des cas. La seconde technique corrige ce phénomène avec une précision de la classification par pixel attenant les $99\%$ sur des images à faible résolution. Elle présente également l'avantage de nécessiter un moindre temps de prédiction. Toutefois, ces deux méthodes se révèlent très sensibles à la présence d'implants et aux fortes déformations du rachis cervical.
        \\

        Certains travaux permettent également l'extraction des régions vertébrales thoraciques et lombaires dans des vues frontales, et méritent donc d'être citées puisque peu de recherche sur les cervicales sont réalisées en vue frontale.

        A ce titre, \cite{Ebrahimi2018} propose une méthode de détection des pédicules des vertèbres thoraciques et lombaires en vue frontale et utilise la méthode de \cite{Humbert2009} pour restreindre la zone de recherche. La méthodologie semi-automatique proposée par {\itshape Humbert et al.} consiste en l'application de deux modèles paramétriques couplés : un pour la forme globale de la colonne vertébrale, et un second pour la forme vertébrale. Ces modèles sont respectivement modifiés par une inférence statistique \og longitudinale \fg et une \og transversale \fg pour affiner la forme de la colonne.

        Cette recherche est poursuivie par \cite{Aubert2016} qui propose de régresser les centres des vertèbres thoraciques et lombaires avec un réseau de neurones profond, et initialise ensuite un modèle statistique de forme ({\itshape SSM}) qui s'affine pour décrire la colonne vertébrale. Cette méthode automatise la précédente en restant dans l'utilisation de modèles statistiques représentant le rachis.
        \\

        Ces différentes méthodes ont un intérêt fort pour différentes recherches puisqu'en réduisant la zone d'intérêt à la région cervicale, les traitements ultérieurs voient leur complexité de calcul réduite. Toutefois, une majorité des recherches assimilent la détection des vertèbres à celle de certains de leurs points de repère ; ces recherches sont étudiées dans la sous-section suivante.
        \\\\

    \subsection{Localisation des marqueurs cervicaux}

        En terme de détection des vertèbres cervicales, l'état de l'art se concentrait initialement sur la localisation de marqueurs anatomiques, et une majorité des recherches abordent cette problématique.
        \\

        Une application de la transformée généralisée de Hough permit par exemple à \cite{Lahrmam2012} de régresser les centres vertébraux des cervicales C3 à C7 sur des radiographies en coupe sagittale. L'étude exhibe un taux de détection à $96.5\%$, supérieur à l'état de l'art de la localisation de centre vertébraux qui se concentrait alors sur les lombaires.

        Cette technique se valorise en permettant notamment l'automatisation de méthodes d'extraction de la région cervicale comme celle de \cite{Benjelloun2006a}.
        \\

        Le diagnostique de la mobilité cervicale nécessitant simplement la mesure des orientations vertébrales, la détection des coins vertébraux la rend possible. Dans cette optique, \cite{Mahmoudi2007} propose d'isoler les cervicales C3 à C7 dans des quadrilatéres selon \cite{Benjelloun2006a} pour y appliquer un détecteur de coins de {\itshape Harris} qui fournit un set de pixels candidats en tant que coins vertébraux. Ces candidats sont ensuite filtrés en fonction de leur proximité les uns des autres et des valeurs qui leur sont associés par la fonction de réponse du détecteur de {\itshape Harris}.

        Cette étude rend compte de la difficulté qu'implique l'utilisation de ces détecteurs : un nombre trop important de candidats sont proposés. Pour pallier à ce problème, ce papier propose également une étape de prétraitement du contraste de l'image médicale afin de rendre les bords des vertèbres davantage visibles.

        Inspiré par cette méthodologie, \cite{Benjelloun2009} propose l'extraction des coins antérieurs des vertèbres C3 à C7 à parti d'un unique clique sur un centre vertébral. Une zone de recherche se dessine autour de ce clique et un détecteur de {\itshape Harris} est appliqué. Les deux premiers coins sont alors sélectionnés par rapport à leurs distances au clique et à la valeur du gradient de l'image en ses points. Un traitement itératif est alors lancé, traçant une parabole passant par les coins déjà détectés afin de délimiter les zones de recherche supérieure et inférieure des deux coins suivants par application répétée des détecteurs de {\itshape Harris} et de la sélection des meilleurs candidats. L'article présent enfin une inférence statistique visant à estimer les coins non détectés par la méthode proposée.

        Malgré les divers processus de prétraitement de contraste proposés, les filtres de pixels candidats à coins vertébraux et les post-traitements permettant une estimation statistique des coins non détectés, les techniques basées sur l'utilisation des détecteurs de {\itshape Harris} sont difficilement améliorables. En effet, ces outils sont très sensibles au bruitage présent en réseau de la superposition des os, et sont ainsi amener à fournir une trop grande abondance de pixels candidats ou de ne pas détecter certains marqueurs.
        \\

        Afin de proposer une solution alternative, \cite{Lecron2010} présente une utilisation détournée de l'approximation polygonale de contours, généralement utilisée pour approcher une forme géométrique à partir d'un ensemble de points, pour détecter les quatre coins vertébraux en adaptant un quadrilatère à la forme de chaque cervicale ciblée. Cette méthode nécessite un pré-traitement sur le contraste et un filtrage de {\itshape Canny} afin de mettre en évidence les contours osseux présents dans la radiographie.

        \cite{Lecron2010} se compare à \cite{Mahmoudi2007} afin de montrer sa fiabilité quant à l'analyse de la mobilité cervicale, et obtient des résultats satisfaisant avec des écarts angulaires inférieurs à ceux de la méthode initiale de {\itshape Mahmoudi et al.} par rapport à des mesures réalisées par des experts. L'avantage par rapport au détecteur de coins de {\itshape Harris} consiste principalement en la réduction du nombre d'hyperparamètres et de la sensibilité de la détection au bruit.

        Cette technique se complète avec \cite{Lecron2012b, Lecron2012c} qui étudient l'identification des coins cervicaux antérieurs détectés par approximation polygonale. Pour chaque coin antérieur, les descripteurs {\itshape SIFT} et {\itshape SURF} sont utilisés en fonction de l'article pour générer des caractéristiques visuelles qui permettent à un {\itshape SVM} pré-entrainé de classer ces coins dans les catégories antérieurs supérieurs ou inférieurs.

        \cite{Lecron2012a} rapporte que l'utilisation des descripteurs {\itshape SURF} est la plus adaptée pour les cervicales C3 à C6, mais qu'à l'inverse {\itshape SIFT} fonctionne davantage pour la C7. De plus, il propose un processus de correction en inférant les coins non détectés en prenant en compte les alternances antérieur inférieur/supérieur. La classification des coins est précise à $92.5\%$, et ils sont détectés avec une distance euclidienne moyenne de $2.9 mm$ entre les points prédits et les annotations.
        \\

        Des techniques d'apprentissage automatique sont également utilisées à ces fins dans la littérature. \cite{AlArif2015a, AlArif2015b} présentent ainsi une procédure d'estimation de la position des coins vertébraux par une forêt aléatoire de {\itshape Hough} établissant sa prédiction à partir les caractéristiques pseudo-\textit{Haar} de patchs variés extraits de l'image médicale.

        Dans la continuité de l'utilisation d'une forêt aléatoire de {\itshape Hough}, \cite{AlArif2017c} propose d'entrainer le modèle sur des caractéristiques visuelles contextuelles et de le rendre invariant aux changements d'échelle et aux rotations en l'entrainé sur des patchs de propriétés diverses. La forêt aléatoire est entrainée sur un ensemble de données particuliérement difficile, la rendant ainsi robuste, puisqu'il contient des radiographies de résolutions, échelles et qualités variées ainsi que certaines présentant de sévéres troubles cervicaux ou des implants. Il exhibe une distance moyenne point à point entre les prédictions et les données de validation de $2.3 mm$ et met également en évidence que ces résultats sont équivalents aux annotations manuelles d'un expert, puisque celles-ci sont sujettes à l'interprétation.

        Les réseaux de neurones sont également mis à l'honneur pour la localisation de marqueurs anatomiques cervicaux en coupe sagittale. \cite{Slabaugh2017} propose l'entrainement d'un réseau de neurones totalement convolutifs sur des distributions de probabilités d'être un pixel représentant un coin vertébral. L'idée cachée derrière cet entrainement est qu'il n'existe pas de marqueur visible défini pour les coins vertébraux dont les angles sont arrondis ; une telle distribution en tant qu'annotation d'entrainement permet de neutraliser davantage la subjectivité de l'interprétation de l'expert ayant annoté la base de données.

        Dans ce même papier, {\itshape Slabaugh et al.} compare les performances de ce réseau de neurones aux forêts aléatoires de {\itshape Hough} sur les mêmes données challengeantes ; le premier modèle présente une distance euclidienne moyenne de $1.5 mm$ entre les prédiction sur les données de validation et leurs annotations, et le second $2.5 mm$. Le réseau de neurones fournit ainsi une approche rapide et efficace pour la détection des marqueurs anatomiques. Toutefois, ce modèle reste sensible à la présence d'implants ou à une trop forte pathologie cervicale.
        \\

        La localisation des vertèbres selon certains marqueurs peut donc suffir à l'analyse de la mobilité des patients ; mais pour une reconstruction {\itshape 3D}, les délimitations des cervicales forment une information de première valeur. La section suivante de cette revue de la littérature se concentre ainsi sur la segmentation des cervicales dans des radiographies.
        \\\\


\section{Segmentation vertébrale}

    La littérature propose différentes approches de segmentation dans le cadre du traitement d'image médicale.

    Ainsi, \cite{Benjelloun2006b, Benjelloun2007} proposent par exemple une détection des bords vertébraux basée sur une évaluation de la signature polaire et du gradient d'intensité afin d'extraire un ensemble de points appartenant au contour de la vertèbre ; puis un assemblement de splines cubiques reforme approximativement le contour de la vertèbre.

    \cite{Benjelloun2008} compare cette méthode initiale à un modèle dynamique de contours discrets, proposé par \cite{Lobregt1995}, qui après initialisation adapte sa forme en analysant les propriétés locales de l'image médicale comme des forces internes et externes qui font converger les points de contrôle du modèle vers leurs positions finales sur le contour.
    \\

    La majorité des approches de segmentation de corps osseux dans une radiographie se distinguent principalement en deux catégories :

    \begin{enumerate}
        \item Premièrement, les approches basées sur les dérivées du modèle statistique de formes, comme l'{\itshape Active Shape Model} ({\itshape ASM}) initialement proposé par \cite{Cootes1995}, exhibent de bonnes performances car elles tiennentt compte de la forme globale du coprs segmenté.
        \item Avec le succés grandissant de l'apprentissage automatique en analyse d'images médicales, des méthodes de classification de pixels ou de régression de caractéristiques de formes sont mises en place.
        \\
    \end{enumerate}

    \subsection{Approche statistique}

        L'application des modèles statistiques de forme pour la segmentation des vertèbres C3 à C7 en vue sagittale fut proposée par \cite{Benjelloun2010}. Un modèle de forme moyen {\itshape ASM} est généré à l'entrainement. Une recherche est initialisée pour chaque cervicale cible grâce aux cliques d'un opérateur sur ses coins antérieurs. Cette étude démontre l'efficacité de l'{\itshape ASM} dans le cadre de la segmentation des vertèbres cervicales : en moyenne, $93\%$ des surfaces vertébrales sont segmentées.
        \\

        \cite{Lecron2010} propose d'automatiser cette approche en se passant des cliques d'un opérateur pour l'initialisation. En lieu et place, les coins antérieurs des vertèbres C3 à C7 sont détectés automatiquement par approximation polygonale, telle que décrite dans la section précédente. Cette technique permet, accompagner d'une régression complétant les coins non détectés, une détection des coins satisfaisante afin que les initialisations des modèles vertébraux {\itshape ASM} soient proches des vertèbres cibles et réussissent au mieux la segmentation.

        Cette méthodologie permet à cette équipe de proposer un {\itshape framework} d'analyse automatique des vertèbres cervicales C3 à C7 dans des radiographies biplanaires. L'utilisation de l'approximation polygonale pour détecter les coins antérieurs des vertèbres permet une initialisation du modèle statistique proche de la cervicale cible. Ainsi, la recherche {\itshape ASM} est lancée dans les conditions adéquates à sa convergence vers la forme vertébrale voulue

        \cite{Mahmoudi2010} propose alors d'améliorer la méthodologie de \cite{Lecron2010} selon un gain de coût temporel. En effet, en raison du filtrage de {\itshape Canny}, qui représente un processus coûteux de traitement de l'image, l'initialisation du modèle {\itshape ASM} est une étape exigente en terme calculatoire. La proposition est ainsi d'utiliser une architecture mutlti-coeurs associant des processeurs et des cartes graphiques afin de paralléliser le filtrage de {\itshape Canny}.

        Cette idée fut reprise dans \cite{Lecron2011} qui démontre qu'une architecture combinant quatre coeurs de processeurs graphiques et huit coeurs classiques détecte et segmente les cervicales selon cette méthodologie vingt fois plus vite qu'un simple processeur.



    \subsection{Approche par apprentissage automatique}

\section{Identification vertébrale}

\section{Méthodologie proposée}
