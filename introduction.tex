Les traumatismes de la colonne vertébrale atteignent une large partie des populations de toute origine ; et le risque d'en souffrir augmente avec l'âge. Les conséquences de ces troubles sur la santé sont variables. Un patient peut aussi bien souffrir de légères douleurs l'handicappant, perdre sa mobilité, ou bien voir son diagnostique vital mis en jeu dans les cas les plus extrèmes.

Nécessitant généralement un suivi régulier, le traîtement de ces pathologies implique des coûts élevés imputables aux hopitaux, aux assurances ou parfois aux particuliers. De plus, certaines des déformations de la colonne vertébrale nécessitent des interventions chirurgicales intrusives, comme la pose d'implants. Les professionnels de la santé ont a fortiori besoin de visualiser l'état des membres spinaux de leurs patients afin d'être en mesure de diagnostiquer et traîter leurs troubles.
\\

La colonne vertébrale est une structure anatomique tridimensionnelle complexe. Il est donc nécessaire d'observer cette structure dans un domaine {\itshape 3D} ; nécessité se vérifiant davantage pour de sévères déformations telles que la scoliose idopathique.


Les dispositifs d'imagerie médicale les plus utilisés à ce jour pour parfaire à ce besoin sont les imageries tomodensitométriques et par résonance magnétique. Les premières permettent une visualisation {\itshape 3D} précise des structures osseuses, alors que les secondes permettent un diagnostique fin des tissus mous.

Mis en comparaison avec d'autres systèmes de capture tels que les radiographies à rayons X, ces dispositifs présentent les désavantages du coût élevé et des expositions fortes aux radiations.
\\

C'est à ce titre que l'état de l'art tente de les remplacer par des dispositifs de radiographies biplanaires, permettant la capture simultanée de radiographies du patient dans les vues sagittale et frontale. L'idée fondamentale de ce système est d'utiliser les informations présentes dans les deux vues radiographiques afin de reconstruire les structures osseuses analysées en {\itshape 3D} à l'aide de principes de stéréo-correspondance entre les deux vues capturées.
\\

Un expert requiert un temps important pour annoter entièrement des radiographies avec les marqueurs anatomiques conduisant à la reconstruction tridimensionnelle. L'automatisation de l'analyse des radiographies frontales et sagittales, pour en extraire les informations requises à la reconstruction des corps osseux, relève alors d'un intérêt premier à ce système de capture.

Ainsi, l'automatisation de l'analyse vertébrale, et particulièrement des cervicales dont l'état impacte directement la mobilité des patients, est un sujet de recherche sur lequel différentes équipes se sont penchées.

Dans ce contexte, le {\itshape Laboratoire de recherche en Imagerie et Orthopédie} (LIO) a initié un projet de recherche sur l'automatisation de la détection et de la segmentation vertébrale, et de l'identification du niveau vertébral, dans des radiographies biplanaires à faibles doses de radiation capturées par le dispositif d'{\itshape ÉOS Imaging\textregistered}.
\\

S'inscrivant dans le cadre de ce projet, la recherche détaillée dans ce mémoire se concentre sur l'automatisation de l'analyse des vertèbres cervicales. L'objectif explicite de cette recherche est de proposer une méthode automatique de détection des cervicales, d'extraction de leurs contours dans les radiographies fournies, et d'attribuer à chaque vertèbre son niveau vertébral.

%===================== MODIFY HERE : TO DO
Le présent rapport consiste en une étape vers le livrable final, et résume la littérature scientifique étudiée pour mettre en contexte ce projet par rapport à l'état de l'art.
